\documentclass[a4paper]{letter}
\usepackage[utf8]{inputenc}
\usepackage[swedish]{babel}
\longindentation=0pt

\begin{document}
\signature{Gertrud Kleberg \hspace{0.9cm} Linus Nordberg}
\begin{letter}{Förskola och Fritid \\ Farsta stadsdelsförvaltning \\ Box 113 \\ 123 22 FARSTA}.
\address{Gertrud Kleberg, Linus Nordberg \\ Grumsgatan 2 \\ 123 44 FARSTA}

\opening{} 

Vi ansöker om förtur för vår familjehemsplacerade pojke Mustapha
Jafuneh till förskolan Skogsgläntan i Farsta.

Då Mustapha har haft stor omsorgsbrist i sitt liv så har han inte
utvecklats i takt med andra barn vad gäller vissa moment som är
viktiga för fungerande samspel med andra barn. I vissa samspel är han
i takt med sina jämåriga, men i andra så är han betydligt efter. Han
har enligt läkare och psykologer inte några neuropsykiatriska
funktionsnedsättningar utan hans eftersatta utveckling beror på den
omsorgsbrist och de många separationer han fått utstå under sina
första levnadsår.

Det sätt som förskolorna vanligen åldersindelar i Stockholm gynnar
inte Mustaphas utveckling, då man ofta har grupper på 1-3 år och 4-6
år. Då Förskolan Skogsgläntan har en barngrupp med blandade åldrar 1-6
år, ser vi att det är det bästa för Mustapha. Det anser även den
familjebehandlare och den barnhandläggare som arbetar med Mustapha och
oss som familjehemsföräldrar. Vi har sökt men ej hittat andra
förskolor i vår närhet som kan erbjuda denna typ av åldersintegrerad
verksamhet.

Enligt Skolverket så ska barn med Mustaphas förutsättningar få
särskilt stöd i sin utveckling och plats ska erbjudas skyndsamt.Vi
förstår att en plats på Skogsgläntan inte bara kan ``skapas'' för
Mustapha, men vi ser inte heller hur hans behov ska kunna tillgodoses
på de andra förskolor som vi varit i kontakt med.

Bifogar socialt utlåtande från Rinkeby-Kista sdf.

\closing{Med vänliga hälsningar}

\end{letter}
\end{document}
